%Kommas werden am Ende jeder Definition automatisch hinzugefügt.

\newglossaryentry{muss}{ name={muss}, description={
    Diese Funktion ist verbindlich
}}
\newglossaryentry{sollte}{ name={sollte}, description={
    Diese Funktion ist optional und wird implentiert, wenn die notwendigen Ressourcen zur Verfügung stehen
}}
\newglossaryentry{balken}{ name=Balken, description={
    Das einzige Spielelement, welches der Spieler aktiv kontrollieren kann. Der horizontale Balken ist am unteren Bereich des Spielfelds positioniert und der Ball prallt von diesem ab
}}
\newglossaryentry{ball}{ name=Ball, description={
    Das zentrale Spielelement, welches vom Balken und von allen Rändern des Spielfelds abprallt und durch die physikalische
    Interaktion mit dem Balken indirekt vom Spieler gesteuert wird
}}
\newglossaryentry{beta}{name={Beta-Version},description={
    Eine vorläufige und funktionstüchtige Testversion des Spiels, die der Erhaltung von Feedback dient und diesem dynamisch angepasst wird
}}
\newglossaryentry{block}{ name=Block, description={
    Ein Spielobjekt, welches im Invasion-Mode vom Ball zerstört werden muss
}}
\newglossaryentry{classicMode}{ name=Classic-Mode, description={
    Der klassische PONG-Spielmodus, bei dem nur ein Ball und ein Balken auf dem Spielfeld existiert
}}
\newglossaryentry{coin}{ name=Coin, description={
    Die Währung des Spiels
}}
\newglossaryentry{invasionMode}{ name=Invasion-Mode, description={
    Ein erweiterter Spielmodus, bei dem der Ball zusätzlich Block-Objekte zerstört
}}
\newglossaryentry{ladezeiten}{ name=Ladezeiten, description={
    Die Wartezeit beim initialen Laden der App (Starten aus dem Betriebssystem) und Wechsel zwischen Menüs
}}
\newglossaryentry{leben}{ name=Leben, description={
    Eine abstrakte Währung, die der Beendigung eines Spieldurchlaufs dient. Besitzt der Spieler keine Leben mehr, endet das Spiel
}}
\newglossaryentry{niceToHave}{ name=Nice-To-Have, description={
    Optionale Anforderungen, die nicht rechtlich bindend und deren Implementierung völlig dem Entwicklerteam überlassen sind
}}
\newglossaryentry{powerup}{ name=Power-Up, description={
    Ein temporärer Effekt, der z.B. Spielelemente wie den Ball oder den Balken betrifft und deren Verhalten ändert
}}
\newglossaryentry{release}{name={Release-Version},description={
    Die finale und ausgelieferte Version des Spiels
}}
\newglossaryentry{tail}{ name=Schweif, description={
    Kosmetikeffekt hinter dem Ball, welcher die Bahnkurve anzeigt (gleicht dem Schweif einer Sternschnuppe)
}}
\newglossaryentry{point}{ name=Score-Point, description={
    Ein Wert, der dem Spieler als Metrik für seine Fähigkeiten dient. Im Verlauf des Spiels sammelt der Spieler zwangsläufig mehr
}}
\newglossaryentry{skin}{ name=Skin, description={
    Kosmetikeffekte wie verschiedene Sprites/Texturen für Ball, Balken und Spielhintergrund.
    Diese können im Store gegen Coins eingetauscht werden
}}
\newglossaryentry{spieler}{ name=Spieler, description={
    Jede Person, die die Anwendung nutzt \textit{Synonyme: Benutzer, Anwender}
}}
\newglossaryentry{spielfeld}{ name=Spielfeld, description={
    Umfasst den Bereich, in dem das Spiel stattfindet Vertikale Abgrenzungen sind der linke und rechte Bildschirmrand.
    Der obere Bildschirmrand bildet die obere Abgrenzung, die untere Abgrenzung erfolgt auf der Höhe, auf der sich der
    Balken befindet
}}
\newglossaryentry{Top10}{ name=Top-10, description={
    Die zehn besten Scores, die lokal auf einem Gerät (seit Installation) erreicht wurden
}}
\newglossaryentry{achievements}{ name=Achievements, description={
    Errungenschaft, die der Spieler im Laufe des Spiels freischalten kann, welche weder funktionelle noch kosmetische Auswirkungen auf das Spielgeschehen haben
}}
\newglossaryentry{kann}{ name=kann, description={
    Benennt eine optionale Anforderung (siehe "Nice-To-Have")
}}