Der \gls{spieler} verwendet das Spiel PONG hauptsächlich als Gelegenheitsspiel, welches z.B. in Pausen die Langeweile vertreibt. 
Des Weiteren kann der \gls{spieler} das Spiel verwenden, um sich mit anderen durch Top-Scores zu messen oder um der Realität zu entfliehen und Spaß zu haben. 
Der gesellschaftliche Effekt des Austauschs von Highscores und bereits freigeschalteten \glspl{skin} ist ebenfalls ein zentraler Punkt. Die stetige Verbesserung der eigenen Leistung und der dadurch entstehende Ehrgeiz ist ein weiterer Faktor, welcher die \gls{spieler} motiviert.
\\
PONG ist kostenlos erhältlich und kann von Menschen aus allen Lebens\-bereichen genossen werden.

