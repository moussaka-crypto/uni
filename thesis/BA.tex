    % diese Vorlage wurde 2011 von Tobias Banaszak erstellt. post@tobiasb.de
% sie wurde im Juni 2012 etwas angepasst und erweitert.
%
\documentclass[
%a5paper,						% alle weiteren Papierformat einstellbar
%landscape,						% Querformat
%10pt,							% Schriftgröße (12pt, 11pt (Standard))
12pt,							% Schriftgre (12pt, 11pt (Standard))
oneside,							% einseitiges Layout
%twocolumn,						% zweispaltiger Satz
openany,							% Kapitel können auch auf linken Seiten beginnen
headsepline,						% Trennline zum Seitenkopf	
%footsepline,					% Trennline zum Seitenfuß
%chapterprefix,					% vor Kapitelberschrift wird "Kapitel Nummer" gesetzt
%appendixprefix,					% Anhang wird "Anhang" vor die Überschrift gesetzt 
listof=flat,						% setzt z.B. das Abbvz neu, falls die Nummern in die Beschreibung rein ragt
%draft,							% überlange Zeilen in Ausgabe kennzeichnen, setzt keine Bilder un Listings.
numbers=noenddot					% macht aus der Überschrift 1.1. den letzten Punkt weg
]{scrbook}

\pagestyle{headings}				% lebender Kolumnentitel  
\newcommand{\n}{\newline}
\usepackage{setspace}
\onehalfspacing					% 1,5facher Zeilenabstand
\usepackage{caption}
%% Deutsche Anpassungen %%%%%%%%%%%%%%%%%%%%%%%%%%%%%%%%%%%%%
\usepackage[english]{babel}
\usepackage[utf8]{inputenc}	% iso 8859-1 (latin-1) als standart
\usepackage[T1]{fontenc}
\usepackage{url}
\usepackage{lipsum}
\bibliographystyle{IEEEtran}
\usepackage[backend=biber,style=ieee,block=ragged]{biblatex}
\addbibresource{BA_Bib.bib}
\usepackage{csquotes}
%% Packages für Grafiken & Abbildungen %%%%%%%%%%%%%%%%%%%%%%
\usepackage{graphicx} 			% Zum Laden von Grafiken
\usepackage{ctable}				% bunte Tabellen (z.B. grau hinterlegte Spalten)
%more table packages
\usepackage[utf8]{inputenc}
\usepackage{tabularx}
\usepackage{booktabs}
\setcounter{biburlucpenalty}{100}
\setcounter{biburllcpenalty}{100}
\newcommand{\comment}[1]{}

%% Bibliographiestil %%%%%%%%%%%%%%%%%%%%%%%%%%%%%%%%%%%%%%%%%%%%%%%%
%\usepackage{natbib} 

%% Sonstiges Layout %%%%%%%%%%%%%%%%%%%%%%%%%%%%%%%%%%%%%%%%%%%%%%%%
\usepackage[absolute]{textpos}	% Bilder absolut positionieren. Für Logo auf Titelseite
\usepackage{float}

% For colors
\usepackage{xcolor}

% For minted
\usepackage{minted}
\setminted{
  breaklines=true,
  linenos,
  fontsize=\footnotesize,
  frame=lines,
  bgcolor=gray!12,
}

% For placing captions outside of float environments
\usepackage{caption}
% Links klickbar machen (sowohl zu Kapiteln/anderen Referenzen, als auch ins Web:
%
\usepackage[bookmarksnumbered,final]{hyperref}
% für den Druck die folgende Zeile nutzen, damit die Links nicht gesetzt werden
%\usepackage[bookmarksnumbered,draft]{hyperref}
\def\UrlBreaks{\do\/\do-}
 %\usepackage{breakurl}			% bricht URLs um, besonders im Literaturverz. für den Druck auskommentieren
								% ist aber nur nötig, wenn man nicht direkt nach PDF setzt (also latex->dvi->ps->pdf)
								
\usepackage{subfigure} 			%mehrere Bilder in einer Figure
 
%für den pdf-Reader
\hypersetup{
	pdfpagelayout=TwoPageRight, %erste Seite ist Titelseite
	pdfstartview=Fit,
	pdfauthor={}, 
pdftitle={},
pdfsubject={Bachelorarbeit},
pdfkeywords={Bachelorarbeit}
}

\usepackage[final]{listings}		%Quellcode-Listings. final, damit die auch bei draft gesetzt werden
\usepackage{color}					%farben
%\definecolor{DarkGrey}{cmyk}{0,0,0,90}
\definecolor{mint}{rgb}{0, 0.71, 0.678}




% für \floarbarrier, damit man floating-objekte aufhalten kann
% (z.b. wenn man vor einer neuen section alle floats setzen lassen möchte)
\usepackage{placeins}

%um das Logo absolut zu Posistionieren
\setlength{\TPHorizModule}{1mm}
\setlength{\TPVertModule}{\TPHorizModule}
\textblockorigin{0mm}{0mm}

% Absatzeinzug verkleinern
\setlength{\parindent}{0cm} 
% abkürzungsverzeichnis
\usepackage[smaller]{acronym} 
 
% manuelle Worttrennung
\hyphenation{Ad-mi-nis-tra-tor Get-Re-quest Get-Re-sponse  Get-Next-Request Get-Bulk-Request Pro-gram-mie-rung fast-ether-net hexa-de-zi-mal-en}
% im Text kann man mit \- Latex anzeigen, wo es trennen kann: Ein\-horn\-schei\.ße

% um das Beispieldokument mit Inhalt zu füllen
\usepackage{blindtext}


%BOXSTYLE
\usepackage{minted}
\usemintedstyle{vs}

\usepackage{pdfpages}
\begin{document}



%standartumgebung für codelistings setzen
\lstset{showstringspaces=false,frame=lines,breaklines=true,language=[Sharp]C,numberbychapter=true,captionpos=b,prebreak={\Righttorque},basicstyle=\footnotesize}




%%%%%%%%%%%%%%%%%%%%%%%%%%%%%%%%%%%%%%%%%%%%%%%%%%%%%%%%%%%%%%%%%%%%%%%
%
\mainmatter						% Hauptteil
\begin{titlepage}
\thispagestyle{empty}%
\setlength{\oddsidemargin}{0cm}%
\enlargethispage{\baselineskip}
\begin{textblock}{16}(188,8)%
    \includegraphics[width=1.5cm]{Images/fh_logo_rechts.png}%
 \end{textblock}%
\vspace*{-1.0cm}
\LARGE\textbf{FH Aachen}\\
\Large\textbf{Department of Electrical Engineering and Information Technology}\\
\large\textbf{IT Security Subject Area}\\
\large Prof. Dr. Georg Neugebauer
\vspace{2cm}
\begin{center}
	\LARGE\textbf{Bachelor's Thesis}\\
	\vspace{1.5cm}
	\LARGE Zero Trust Security Architectures in Smart Home Environments\\
	\large
	\vspace{1.5cm}
	Submitted by:\\
	Hristomir Dimov\\
	Matriculation number: 3536320\\
	\vspace{1.5cm}
	Study Program: Computer Science\\
	\vspace{1.5cm}
	\today\\
	\vspace{1.5cm}
	Supervisor: Prof. Dr. Georg Neugebauer\\
	Co-supervisor: M.Eng. Sacha Hack 
\end{center}
\end{titlepage}
\pagebreak
%Erklärung
\chapter*{Eidesstattliche Erklärung}
Ich versichere hiermit, dass ich die vorliegende Arbeit selbständig
verfasst und keine anderen als die im Literaturverzeichnis
angegebenen Quellen benutzt habe.\newline

Stellen, die wörtlich oder sinngemäß aus veröffentlichten oder noch
nicht veröffentlichten Quellen entnommen sind, sind als solche
kenntlich gemacht.\newline

Die Zeichnungen oder Abbildungen in dieser Arbeit sind von mir selbst
erstellt worden oder mit einem entsprechenden Quellennachweis
versehen.\newline

Diese Arbeit ist in gleicher oder ähnlicher Form noch bei keiner
anderen Prüfungsbehörde eingereicht worden.
\newline\newline\newline

\par
Aachen, \today
\thispagestyle{empty} %keine Seitenzahlen
\pagebreak
%%%%%%%%%%%%%%%%%%%%%%%%%%%%%%%%%%%%%%

\thispagestyle{empty} 
\section*{Abstract}
\thispagestyle{empty} 
With the growing adoption of Internet of Things (IoT) technology and its essential role in the realization of the smart home, the threat of the end user privacy being compromised is consequently ever more prevalent due to the expanded network surface. The common network architecture ensures the security of the network perimeters, while implicitly trusting that communication within the smart home, typically between IoT devices and the users, is secure, which serves to extend the attack surface of the home network.\\
This bachelor's thesis focuses on the integration of Zero Trust security principles in the smart home network architecture with the intention of providing better security for users. This is achieved by introducing two smart home architectures — one with an on-premise hosting concept and another that relies on the AWS cloud service provider for its hosting process. In that regard, with the combination of the features from both the Zero Trust (multi-factor authentication, monitoring, and analytics) and perimeter-based (IP-based traffic filtering) models, two smart home solutions established on the architectures have been realized.\\
Furthermore, this work evaluates the two smart home environments with the help of predefined primary and secondary metrics, aiming to assess and formulate an answer to the question on whether a Zero Trust smart home network should be hosted locally or in the cloud, depending on the user's needs for security and technical expertise.

\vspace*{1.0cm}
\textbf{Keywords:} Zero Trust, IoT, Smart Home, Architectures, Security, On-premise, Cloud, Multi-factor Authentication, Metrics
%% Erzeugung von Verzeichnissen %%%%%%%%%%%%%%%%%%%%%%%%%%%%%%%%%%%%%%%
%\renewcommand{\listfigurename}{List of Figures}
\vspace{-0.5cm}\tableofcontents			% Inhaltsverzeichnis
\listoftables				% Tabellenverzeichnis
\listoffigures				% Abbildungsverzeichnis
%\lstlistoflistings
%\printnomenclature			% Abkürzungsverzeichnis
\pagebreak
\vspace{-0.5cm}\chapter*{List of Abbreviations}
\markright{List of Abbreviations} 
\begin{acronym}[Acronyms]
    \acro{AMI}{Amazon Machine Image}
    \acro{AWS}{Amazon Web Services}
    \acro{BLE}{Bluetooth Low Energy}
    \acro{CLI}{Command-line interface}
    \acro{CSP}{Cloud Service Provider}
    \acro{EC2}{Elastic Compute Cloud}
    \acro{HA}{Home Automation}
    \acro{IaC}{Infrastructure-as-Code}
    \acro{IaaS}{Infrastructure-as-a-Service}
    \acro{IIoT}{Industrial Internet of Things}
    \acro{IoT}{Internet of Things}
    \acro{KPI}{Key Performance Indicator}
    \acro{LAN}{Local Area Network}
    \acro{LR-WPAN}{Low-Rate Wireless Personal Area Networks}
    \acro{MAC}{Media Access Control}
    \acro{MFA}{Multi-Factor Authentication}
    \acro{MQTT}{Message Queuing Telemetry Transport}
    \acro{NIST}{National Institute of Standards and Technology}
    \acro{OS}{Operating System}
    \acro{QR}{Quick-repsonse}
    \acro{RBAC}{Role-based Access Control}
    \acro{RFID}{Radio-Frequency Identification}
    \acro{RPI}{Raspberry Pi}
    \acro{RTT}{Round-trip Time}
    \acro{SBC}{Single-board Computer}
    \acro{SHS}{Smart Home System}
    \acro{SIEM}{Security Information and Event Management}
    \acro{SSH}{Secure Shell}
    \acro{TOTP}{Time-based one-time password}
    \acro{TVOC}{Total Volatile Organic Compounds}
    \acro{VM}{Virtual Machine}
    \acro{VPC}{Virtual Private Cloud}
    \acro{UEBA}{User and Entity Behavior Analytics}
    \acro{ZHA}{Zigbee Home Automation}
    \acro{ZT}{Zero Trust}
    \acro{ZTA}{Zero Trust Architecture}
    \acro{ZTNA}{Zero Trust Network Access}
\end{acronym}
%
%
\chapter{Introduction}
\label{chap:Intro}
    \input{Chapters/Kapitel_1.tex}
\chapter{Fundamentals}
\label{chap:fundamentals}
    \input{Chapters/Kapitel_2.tex}
%% weitere Chapters wird in \chapter{Einleitung}	hinzufügen
\chapter{Theoretical Part}
\label{chap:theory}
    \input{Chapters/Kapitel_3.tex}

\chapter{Practical Part}
\label{chap:practical}
    \input{Chapters/Kapitel_4.tex}

\chapter{Evaluation}
\label{chap:evaluation}
    \input{Chapters/Kapitel_5.tex}
    
\chapter{Summary and Outlook}
\label{chap:conclusion}
    \input{Chapters/Kapitel_6.tex}
\newpage    
\appendix
\chapter{Appendix}
    \label{chap:appendix}
    \begin{center}
  \label{fig:local_config}
\end{center}
\inputminted[breaklines, linenos]{yaml}{Code/ha_config/configuration.yaml}
\captionof{figure}{Local Home Assistant configuration.yaml File}

\begin{center}
  \label{fig:cloud_config}
\end{center}
\inputminted[breaklines, linenos]{yaml}{Code/ha_config/cloud_configuration.yaml}
\captionof{figure}{Cloud Home Assistant configuration.yaml File}

\begin{center}
  \label{fig:variables}
\end{center}
\inputminted[breaklines, linenos]{hcl}{Code/tf_config/variables.tf}
\captionof{figure}{variables.tf Terraform File}

\begin{center}
  \label{fig:tfvars}
\end{center}
\inputminted[breaklines, linenos]{hcl}{Code/tf_config/terraform.tfvars}
\captionof{figure}{terraform.tfvars Terraform File}

\begin{center}
  \label{fig:main}
\end{center}
\inputminted[breaklines, linenos]{hcl}{Code/tf_config/main.tf}
\captionof{figure}{main.tf Terraform File}

\begin{center}
  \label{fig:security}
\end{center}
\inputminted[breaklines, linenos]{hcl}{Code/tf_config/security.tf}
\captionof{figure}{security.tf Terraform File}

\begin{center}
  \label{fig:network}
\end{center}
\inputminted[breaklines, linenos]{hcl}{Code/tf_config/networking.tf}
\captionof{figure}{network.tf Terraform File}

\begin{center}
  \label{fig:ec2setup}
\end{center}
\inputminted[breaklines, linenos]{bash}{Code/tf_config/ec2setup.sh}
\captionof{figure}{EC2 Instance Setup Script}


%
%
%
%
%Lieraturverz.
% \printbibliography[title=Bibliography]
% \addcontentsline{toc}{chapter}{Bibliography}
\cleardoublepage
\phantomsection
\addcontentsline{toc}{chapter}{References}
\printbibliography[title={References}]
%
%
\end{document}