\section{Summary}
The purpose of this work was to examine the integration of the Zero Trust security paradigm in two types of smart home environments. As outlined in the first chapter, the two main objectives are derived from the research questions for this topic: first and foremost, to examine how can the ZT security model be integrated into a smart home environment; and secondarily, to provide an answer on whether the superior way to host a Zero smart home system is on premises or in the cloud. In hindsight, both of these objectives have been fulfilled successfully, with regard to addressing the pre-established research questions and providing valuable insights into the topic.\\
Regarding the integration of Zero Trust as part of the first research question, the proposed architectures are based on the setup of three main components: a host device, device gateway and a client device, as depicted in Figure 3.2. Additionally, the concepts for both the on-premise and cloud architectures have outlined as starting configurations with the view to incorporate the most prominent aspects of both the ZT and perimeter-based security principles. In that regard, the local architecture has been conceptualized with the principles of MFA, principle of least privilege, firewall-based remote connection and the tunnel analytics. In contrast to that, the cloud concept incorporates a shorter set of features: MFA, least privilege infrastructure provision and IP-based traffic filtering. Furthermore, for the purpose of evaluating both architecture concepts and their implementations described in Chapter 4, there was a latent need for KPI measurements specifically created for that. Summarized in Table 3.1, they are separated into primary (relating to security) and secondary metrics. After providing a short insight into the limitations and challenges of the architectures in the context of IIoT, as well as for their individual features, the process of practically setting up the smart home solutions derived from them has been described in detail.

The on-premise architecture implementation has utilized a Raspberry Pi booted with the Home Assistant OS, establishing this environment as part of a larger local network. In terms of the configuration settings, they are managed in a central YAML configuration file as part of the file system, used to implement an MFA mechanism using TOTP, IP address banning and a remote tunnel connection. On the subject of the devices within the smart home environment, both Zigbee and BLE devices have been integrated with two separate automations, according to their functionality. Figure 4.14 depicts the full architecture of the on-premise environment, summarizing all its main features.\\
Pertaining to the cloud-based architecture implementation, it has been actualized using an EC2 virtual machine as part of AWS and Terraform to manage its configuration. In addition, the Docker Compose software is used to run the Home Assistant software in a container environment, as opposed to an explicit OS. Figure 4.15 depicts the full architecture of the cloud-based environment, which provides a summary of its integrated features. The configuration settings as well as the fundamental Terraform concepts have been outlined, related to their role in the virtual machine setup and utilized to provision the needed network infrastructure. In regard to the security features, MFA has been implemented through the Home Assistant container application, with the IP-based traffic filtering being based on the definition of security groups as part of the VPC settings. The final steps in the setup of the environment have been achieved by simulating user input through a Shell script, aiming to install all the prerequisite packages for the EC2 instance and start the Home Assistant application.\\
Before evaluating the results from their respective Zero Trust smart home architectures, the last part of Chapter 4 provides several key conclusions derived from their practical implementation with reference to the setup processes of both smart home environments and the complexity of their device integration.\\
The evaluation of the thesis results is composed of three parts. It begins with reminding the reader of both research questions and provides a brief overview of the insights in the practical implementations of both smart home environments. Subsequently, the chapter reviews the implementations of both smart home environments in the context of the first research question on whether Zero Trust principles are applicable in a SHS, including the effectiveness of the introduced security measures, usability, and their shortcomings. Moreover, a final verdict for both SHS solutions has been included, referring to their advantages and disadvantages, together with an answer to the second research pointing out the on-premise implementation as more suitable for a Zero Trust SHS. Conclusively, the last part of the evaluation is focused on the primary and secondary evaluation metrics, defined as part of Section 3.2. The estimated results in relation to each of the metrics are summarized in two tables.

To summarize, this thesis work explored the adoption of the Zero Trust security model and its core principles within the context of smart home environments. As part of this process, two smart home architectures have been conceptualized with a combination of Zero Trust and perimeter-based security measures. Additionally, the overall and specific limitations of these architectures have been described, together with evaluation metrics to estimate the security of both environments. Considering the practical implementation of SHS solutions on the basis of the aforementioned architectures, both yielded tangible results with the cloud-based environment being more unsophisticated in its setup, yet the on-premise one has proven more refined in its implementation. This distinction persists in the evaluation of the work, as it not only adopts more of the Zero Trust model's principles, but also functions more effectively as a separate smart home environment, taking into account the metric-based analysis. 

\section{Outlook}
In bringing this thesis work to a close, the conceptualized SHS architectures and their respective implementations yielded intriguing insights into the adoption of Zero Trust security into smart home environments. Consequently, two main focuses with research potential arise from the subject of this work, which could pave the way for the continued development of solutions in this academic niche.\\

The first research direction places emphasis on the concept of a Zero Trust security architecture based on edge computing principles. As described in \cite{edge_comp}, the case study of edge computing for smart homes sets a clear boundary pertaining to the processing of data, mostly limiting this task to the home network with a view towards data privacy. Therefore, this line of research could serve to refine the originally introduced on-premise architecture to keep the access to and control over the data from IoT devices exclusively in the local environment, while also giving users the opportunity to restrict accessibility to highly sensitive data, should it be used by service providers \cite{edge_comp}. In the context of Zero Trust, this process could include implementing RBAC for different user profiles and enforcing it through IAM policies as part of the edge operating system in this architecture, intended to breach the gap between the smart home system and its security measures.\\

Secondly, since the implementations of the introduced architectures amount to starting configurations for ZT in smart homes, they both offer many opportunities for improvement. This is especially true for the cloud-based smart home solution, which, when compared to its on-premise counterpart, lacks in the device integration process. In that regard, AWS offers additional services that can be worth investigating for the integration and monitoring of IoT devices in a ZTA. For instance, AWS IoT Core is responsible for connecting IoT devices and provides authentication measures to other AWS resources through device certificates, whereas the AWS Device Defender service could be used to establish secure baseline behaviour for each of the IoT devices within the network and monitor for anomalous activity \cite{aws_iot_zt}. It would also be beneficial to explore the mitigation approaches these AWS IoT services can adopt for non-compliant device behaviour.

