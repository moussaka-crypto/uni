With the growing adoption of Internet of Things (IoT) technology and its essential role in the realization of the smart home, the threat of the end user privacy being compromised is consequently ever more prevalent due to the expanded network surface. The common network architecture ensures the security of the network perimeters, while implicitly trusting that communication within the smart home, typically between IoT devices and the users, is secure, which serves to extend the attack surface of the home network.\\
This bachelor's thesis focuses on the integration of Zero Trust security principles in the smart home network architecture with the intention of providing better security for users. This is achieved by introducing two smart home architectures — one with an on-premise hosting concept and another that relies on the AWS cloud service provider for its hosting process. In that regard, with the combination of the features from both the Zero Trust (multi-factor authentication, monitoring, and analytics) and perimeter-based (IP-based traffic filtering) models, two smart home solutions established on the architectures have been realized.\\
Furthermore, this work evaluates the two smart home environments with the help of predefined primary and secondary metrics, aiming to assess and formulate an answer to the question on whether a Zero Trust smart home network should be hosted locally or in the cloud, depending on the user's needs for security and technical expertise.

\vspace*{1.0cm}
\textbf{Keywords:} Zero Trust, IoT, Smart Home, Architectures, Security, On-premise, Cloud, Multi-factor Authentication, Metrics