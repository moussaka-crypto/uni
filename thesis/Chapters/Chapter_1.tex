The rise of smart homes related to the surge of Internet of Things (IoT) devices, promoting convenience and control in everyday life \cite{chaurasia_enhanced_2019} has seen a substantial development in the last decade. One of the chief problems that impact smart homes and their networks is the process of ensuring not only the security of external communication, but also within the network itself, with respect to the user's privacy. In the context of this work, the Zero Trust security model, based on the principle of “never trust, always verify”, is intended to oppose the implicit trust placed on the network perimeters \cite{kang_theory_2023} and thus provide more efficient security measures for smart homes. This paper is intended for readers to get acquainted with the topic of integrating Zero Trust for smart homes and home automation, as well as ones researching a way to alleviate the security of a home network.
 
\section{Motivation and Objectives}
Given the fact that in the current age of information technology, the interconnectedness of all kinds of devices from smartphones to different appliances through the Internet has become an expected commodity in people's homes. This phenomenon makes it convenient for consumers to transform parts of or whole living quarters into an IoT-based Smart Home System (SHS), allowing users to automate, remote monitor and control their IoT-integrated living space with different goals in mind ranging from better privacy to higher quality of life through a more intelligent environment \cite{chakraborty_smart_2023}.\\
However, as the home itself is being regarded as a network, this integration of IoT devices is marked by a latent need for their network security. In light of this, a common example is that once a smart device has been paired within the home network, it is assumed to be secure and without any need for authentication of its identity or status. This, in turn, opens a possible backdoor for threat agents to compromise the security of the device and of the smart home itself.\\
To tackle the potential vulnerabilities of a perimeter-oriented network topology, this work proposes and explores the integration process of the Zero Trust security model in a smart home environment. More notably, this thesis aims to give insight, from both a security and a usability point of view, on whether such an environment is more appropriately hosted locally or by utilizing a cloud service provider.
Therefore, the main research questions this bachelor's thesis aims to answer can be formulated, as follows:
\begin{itemize}
    \item How well is the Zero Trust security paradigm integrated into a smart home environment?
    \item Should a Zero Trust smart home network be hosted locally or in the cloud?
\end{itemize}

\section{Structure}
The first chapter introduces the reader to the main topic of the thesis. It also outlines the motivation behind delving into the subject, as well as clarify the research questions and objectives of the bachelor's thesis.\\
The second chapter opens with a comprehensive overview of the origins and core principles behind smart homes and Zero Trust, while later moving on to describe the fundamental technologies and concepts with regard to the implementation of the two smart home architectures and their practical implementations.\\
Following this, the third chapter provides an overview of the theoretical work with the Zero Trust model in the context of smart homes, while subsequently describing the conception of the on-premise and cloud-based ZT security architectures. The chapter also introduces primary and secondary metrics, relating to the security and overall user experience respectively, which have been defined for the context-based comparison of both implementation scenarios. Furthermore, this chapter identifies the limitations of these architectures in a brief overview.\\
The fourth chapter provides a short summary of the features of the implemented SHS solutions, while later explaining in more detail the process for each of the practical implementations. Even though both environments have an initially identical Zero Trust configuration, as part of this work, they deliver unique insights into the setup and device integration process, as outlined in the last section of this chapter.\\
Regarding the analysis of the work, the fifth chapter begins by reminding the user of the main research questions of this thesis and comprises a more in-depth synthesis of the results of this work and a collection of the insights gathered from working on the practical implementation of both environments. Following up on that, this paper provides a general assessment of the implementation of both the locally operated and cloud-operated smart home environment, providing an answer to the aforementioned research directions. Subsequently, concluding the chapter, an assessment based on the evaluation metrics from Chapter 3 is made, estimating the success of both SHS environments.\\
Finally, the last chapter of this work starts with a thorough summary of the whole thesis and its results. Both of the theoretical and the practical findings from each chapter are recapitulated, pointing out the positive and negative aspects. Moreover, the last parts of the chapter provide an outlook on different directions the work could take, regarding the idea of another ZT security architecture based on the concept of edge computing, as well as the further development of the cloud-based SHS solution using a collection of AWS IoT services.